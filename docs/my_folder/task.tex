%%%% Начало оформления заголовка - оставить без изменений !!! %%%%
\input{my_folder/task_settings}	% настройки - начало 
	
				{\normalfont %2020
						\MakeUppercase{\SPbPU}}\\
				\institute\\
				\highSchool

\par}\intervalS% завершает input

				
				

\intervalS{\centering\bfseries%

				ЗАДАНИЕ\\
\intervalS\normalfont%
				на выполнение курсовой работы по дисциплине\\
				<<Технология объектно-ориентированного программирования>>

\par}\intervalS%

студенту \uline{\AuthorFullDat{}} \hfill группа: \uline{~\group}\\
семестр: \uline{3}
\intervalS
%%%%
%%%% Конец оформления заголовка  %%%%
 	
	
	
\begin{enumerate}[1.]
	\item Тема работы: {\expandafter \ulined \thesisTitle.}
	\item Срок сдачи студентом законченной работы: \uline{\thesisDeadline.} 
	\item Исходные данные по работе: \uline{материалы по курсу <<Технология объектно-ориентированного программирования>>.}%
	%\printbibliographyTask % печать списка источников % КОММЕНТИРУЕМ ЕСЛИ НЕ ИСПОЛЬЗУЕТСЯ
	% В СЛУЧАЕ, ЕСЛИ НЕ ИСПОЛЬЗУЕТСЯ МОЖНО ТАКЖЕ ЗАЙТИ В setup.tex и закомментировать \vspace{-0.28\curtextsize}
	\item Содержание работы (перечень подлежащих разработке вопросов):
	\begin{enumerate}[label=\theenumi\arabic*.]
		\item Проектирование приложения для анализа файловой системы.
		\item Реализация функционала.
	\end{enumerate}
	\item Перечень графического материала (с указанием обязательных рисунков): 
	\begin{enumerate}[label=\theenumi\arabic*.]
		\item Диаграмма классов UML.
		\item Диаграммы IDEF0 порядка выполнения программы.
	\end{enumerate}		
	\item Дата выдачи задания: \uline{\thesisStartDate.}
\end{enumerate}

\intervalS%можно удалить пробел

\noindent
\begin{tabularx}{\linewidth}{lXl}
	Cтудент      &    & \Author     \\[\mfloatsep]
	
	Преподаватель     &    & \Supervisor     \\[\mfloatsep]
	
\end{tabularx} %


\input{my_folder/task_settings_restore}	% настройки - конец