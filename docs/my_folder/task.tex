%%%% Начало оформления заголовка - оставить без изменений !!! %%%%
\thispagestyle{empty}%
\setcounter{tskPageFirst}{\value{page}} %сохранили номер первой страницы Задания
\ifnumequal{\value{tskPrint}}{1}{% если двухсторонняя печать Задания, то...
	\newgeometry{twoside,top=2cm,bottom=2cm,left=3cm,right=1cm,headsep=0cm,footskip=0cm}
	\savegeometry{MyTask} %save settings
	\makeatletter % задаём оформление второй страницы ВКР как нечетной, а третьей - как чётной
	\checkoddpage % проверка четности из memoir-класса
	\ifoddpage
	\else
		\let\tmp\oddsidemargin
		\let\oddsidemargin\evensidemargin
		\let\evensidemargin\tmp
		\reversemarginpar
	\fi
	\makeatother
}{} % 
\pagestyle{empty} % удаляем номер страницы на втором/третьем листе
\makeatletter
\newrefcontext[labelprefix={3.}] % задаём префикс для списка литературы
\makeatother
\setlength{\parindent}{0pt}
{\centering\bfseries%
%	\small	% настройки - начало 
	
				{\normalfont %2020
						\MakeUppercase{\SPbPU}}\\
				\institute\\
				\highSchool

\par}\intervalS% завершает input

				
				

\intervalS{\centering\bfseries%

				ЗАДАНИЕ\\
\intervalS\normalfont%
				на выполнение курсовой работы по дисциплине\\
				<<Технология объектно-ориентированного программирования>>

\par}\intervalS%

студенту \uline{\AuthorFullDat{}} \hfill группа: \uline{~\group}\\
семестр: \uline{3}
\intervalS
%%%%
%%%% Конец оформления заголовка  %%%%
 	
	
	
\begin{enumerate}[1.]
	\item Тема работы: {\expandafter \ulined \thesisTitle.}
	\item Срок сдачи студентом законченной работы: \uline{\thesisDeadline.} 
	\item Исходные данные по работе: \uline{материалы по курсу <<Технология объектно-ориентированного программирования>>.}%
	%\printbibliographyTask % печать списка источников % КОММЕНТИРУЕМ ЕСЛИ НЕ ИСПОЛЬЗУЕТСЯ
	% В СЛУЧАЕ, ЕСЛИ НЕ ИСПОЛЬЗУЕТСЯ МОЖНО ТАКЖЕ ЗАЙТИ В setup.tex и закомментировать \vspace{-0.28\curtextsize}
	\item Содержание работы (перечень подлежащих разработке вопросов):
	\begin{enumerate}[label=\theenumi\arabic*.]
		\item Проектирование приложения для анализа файловой системы.
		\item Реализация функционала.
	\end{enumerate}
	\item Перечень графического материала (с указанием обязательных рисунков): 
	\begin{enumerate}[label=\theenumi\arabic*.]
		\item Диаграмма классов UML.
		\item Диаграммы IDEF0 порядка выполнения программы.
	\end{enumerate}		
	\item Дата выдачи задания: \uline{\thesisStartDate.}
\end{enumerate}

\intervalS%можно удалить пробел

\noindent
\begin{tabularx}{\linewidth}{lXl}
	Cтудент      &    & \Author     \\[\mfloatsep]
	
	Преподаватель     &    & \Supervisor     \\[\mfloatsep]
	
\end{tabularx} %


\setcounter{tskPageLast}{\value{page}} %сохранили номер последней страницы Задания
\setcounter{tskPages}{\value{tskPageLast}-\value{tskPageFirst}}
\newrefsection % начинаем новую секцию библиографии
\newrefcontext % удаляем префикс к пунктам списка литературы
\restoregeometry % восстанавливаем настройки страницы
\pagestyle{plain} % удаляем номер страницы на первой/второй странице Задания
\setlength{\parindent}{2.5em} % восстанавливаем абзацный отступ
%% Обязательно закомментировать, если получается один лист в задании:
\ifnumequal{\value{tskPages}}{0}{% Если 1 страница в Задании, то ничего не делать.
}{% Иначе 
% до 2020 года требовалось печатать задание на 1 листе с двух сторон и не подсчитывать вторую страницу
%\setcounter{page}{\value{page}-\value{tskPages}} 	% вычесть значение tskPages при печати более 1 страницы страниц
}%
\AtNextBibliography{\setcounter{citenum}{0}}%обнуляем счетчик библиографии	% настройки - конец