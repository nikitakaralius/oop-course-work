\chapter*{Введение} % * не проставляет номер
\addcontentsline{toc}{chapter}{Введение} % вносим в содержание

Цель работы - спроектировать и разработать приложение для анализа файловой системы, обрабатывающее заданные директории.
Реализованный проект должен содержать следующий функционал:
\begin{itemize}
	\item подсчет количества файлов в директории;
	\item подсчет количества директорий в директории;
	\item подсчет суммарного размера в байтах;
	\item нахождение самых больших файлов (с учетом заданного порога);
	\item нахождение самых новых файлов (для заданной даты);
	\item поиск дубликатов файлов (совпадение по имени, по размеру); \\
\end{itemize}

В программном решении необходимо:
\begin{itemize}
	\item использовать в реализации принципы ООП (инкапсуляция, полиморфизм, наследование);
	\item реализовать обработку некорректного пользовательского ввода;
	\item задействовать структуры данных из стандартной библиотеки STL;
	\item реализовать работу с файловой системой.
\end{itemize}

%% Вспомогательные команды - Additional commands
%\newpage % принудительное начало с новой страницы, использовать только в конце раздела
%\clearpage % осуществляется пакетом <<placeins>> в пределах секций
%\newpage\leavevmode\thispagestyle{empty}\newpage % 100 % начало новой строки