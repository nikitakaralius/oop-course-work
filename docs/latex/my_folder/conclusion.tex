\chapter*{Заключение} \label{ch-conclusion}
\addcontentsline{toc}{chapter}{Заключение}	% в оглавление 

Подводя итоги, в ходе выполнения курсовой работы было спроектировано и реализовано приложение для анализа файловой системы. В результат вошли все пункты из технического задания.

Кроме того, была реализована событийно-ориентированная архитектура с применением CQRS паттерна и конвейера обработки запросов, которая позволила сделать приложение расширяемым и удобным при разработке. Это отлично демонстрирует тот факт, что в пару строк кода удалось реализовать вывод событий на консоль, не изменяя предыдущую логику. Также получилось добавить целый механизм обработки ошибок во всем приложении, создав два файла и добавив одну строчку в \verb|main|.

Основная логика никак не связана ни с пользовательским вводом, ни с консольным интерфейсом, поэтому при необходимости можно легко и просто сделать настольное графическое приложение для анализа файлов. При этом изменения придется внести всего лишь в одном файле main в моменте конфигурации приложения.